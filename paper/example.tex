Context-Free Language Reachability has six varieties~\cite{reps_ilps_1997}:
\begin{itemize}
\item All-Pairs, there exists a path between all pairs of nodes.
\item Single-Source, there exists a path between a selected source and all other nodes.
\item Single-Target, there exists a path between all other nodes to a selected node.
\item Single-Source\&Target, there exists a path between two specific nodes.
\item Multi-Source, there exists a path between multiple sources and all other nodes.
\item Multi-Target, there exists a path between all other nodes to multiple targets.
\end{itemize}

The first limitation of Context-Free Language Reachability (CFLR) is that some analyses require reasoning over multiple paths.
It is not just that we must know whether or not a path exists, but the analysis wants to know about the shared properties of these paths.
In all varieties listed above, the only property of interest is existence.
There is no reasoning about paths themselves.

The second limitation is that the traditional language for CFLR, the Dyck-Language of k numbers of balanced parentheses, only works for analyses that have clear starts and ends.
Analyses which are attempting to solve for those points, their input graph could not be annotated with a Dyck-Language without first solving for those start and end points beforehand.

\begin{figure}
	\includegraphics[width=0.25\textwidth]{figs/CFG_Example.pdf}
	\caption{A Control-Flow Graph for which traditional Context-Free Language Reachability cannot solve Dominance for. 
Basic Block A dominates E \& F, but B only dominates E. C does not dominate anything.}
	\label{fig:cfg}
\end{figure}

%Since CFLR is looking for the existence of a path between vertices of a graph, it is a powerful tool for demand-based analysis.
%But weak for analyses for which no known demand-based algorithm exists, such as Dominance.

Dominance Analysis on Control-Flow Graphs exhibits both of these limitations.
In Figure \ref{fig:cfg}, it is difficult to use parentheses to symbolize dominance. 
Assume that we could know where to place the symbols to denote the beginning of the dominance and the end of dominance for A.
For example, on the edges from A we can place $(_a$ and on the edges into F we can place $)_a$.
If their positions are already known, it follows that dominance is already calculated for A since we know the region at which it dominates and does not.
Annotating the graph for a separate solver to traverse it would then be redundant work.
This is a contradiction that means that we cannot use a balanced language for this analysis.

%Furthermore, there does not exist two edges to place the beginning and end of dominance for Basic Block B.
%Even if we decide to expand our language to also accept any unclosed $(_x$ as accepted words, and annotate all edges with their source parenthesis
%that would result in B dominating F which is not correct.

Applying transformations on the input graph may be the next possible step, but in what way?
The more processing steps done to the input graph before the final solver just bypasses the CFLR limitation by not using CFLR.
