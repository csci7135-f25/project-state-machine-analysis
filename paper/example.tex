Context-Free Language Reachability has six varieties~\cite{reps_ilps_1997}:
\begin{itemize}
\item All-Pairs, there exists a path between all pairs of nodes.
\item Single-Source, there exists a path between a selected source and all other nodes.
\item Single-Target, there exists a path between all other nodes to a selected node.
\item Single-Source\&Target, there exists a path between two specific nodes.
\item Multi-Source, there exists a path between multiple sources and all other nodes.
\item Multi-Target, there exists a path between all other nodes to multiple targets.
\end{itemize}

The first limitation of Context-Free Language Reachability (CFLR) is that some analyses require reasoning over multiple paths.
It is not just that we must know whether or not a path exists, but the analysis wants to know about the shared properties of these paths.
In all varieties listed above, the only property of interest is existence.
There is no reasoning about paths themselves.

The second limitation is that the traditional language for CFLR, the Dyck-Language of k numbers of balanced parenthesis, only works for analyses that have clear starts and ends.
For analyses which are attempting to solve for those points, a graph could not be annotated with a Dyck-Language without first solving for those start and end points beforehand.

Dominance Analysis on Control-Flow Graphs exhibits both of these limitations.
