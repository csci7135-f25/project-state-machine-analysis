%% This is an abbreviated template from http://www.sigplan.org/Resources/Author/.

\documentclass[acmsmall,review,nonacm]{acmart}
\begin{document}

%%
%% The "title" command has an optional parameter,
%% allowing the author to define a "short title" to be used in page headers.
\title{State-Machine Qualia and it's effects on Context-Free Language Reachability}

%%
%% The "author" command and its associated commands are used to define
%% the authors and their affiliations.
%% Of note is the shared affiliation of the first two authors, and the
%% "authornote" and "authornotemark" commands
%% used to denote shared contribution to the research.
\author{Henri Malahieude}
\email{henri.malahieude@colorado.edu}
%%\author{TBD}
%%\email{tbd@colorado.edu}
\affiliation{%
  \institution{University of Colorado Boulder}
  \country{USA}
}


%%
%% The abstract is a short summary of the work to be presented in the
%% article.
\begin{abstract}
	Since Thomas Reps' paper, Program Analysis via Graph Reachability~\cite{}, 
	much work has been devoted to optimizing the reachability solver~\cite{},
	refining the graph construction process~\cite{},
	and finding the ordering of analysis questions with respect to graph reachability ~\cite{}.
	However, there has been no study yet, to our knowledge, of the types of analysis for which Context-Free Language Reachability (CFLR) cannot be deployed in
	the traditional techniques of a Dyck-Language and the Finite State-Machine Reachability Solver.
	For example, dominance of one basic block of the next is not an analysis for which Dyck-Reachability has been employed for, nor is there has there been work for why that is.
	This work seeks to establish the limitations of CFLR for program analysis by drawing a relationship between the languages employed for each question, 
	thus the state-machine required to compute the reachability answer, and the analysis question of interest.
\end{abstract}

%%
%% This command processes the author and affiliation and title
%% information and builds the first part of the formatted document.
\maketitle

\section{Introduction}
\label{sec:intro}
TODO

\section{Overview}
\label{sec:over}
TODO

\section{Motivating Examples}
\label{sec:examples}
TODO

\section{State-Machines for Reachability}
\label{sec:sm}
TODO

\section{Analyses and Languages}
\label{sec:analyses}
TODO

\section{Related Work}
\label{sec:related}
TODO

\section{Conclusion}
\label{sec:conclusion}
TODO

%%
%% The acknowledgments section is defined using the "acks" environment
%% (and NOT an unnumbered section). This ensures the proper
%% identification of the section in the article metadata, and the
%% consistent spelling of the heading.
\begin{acks}
TODO
\end{acks}

%%
%% The next two lines define the bibliography style to be used, and
%% the bibliography file.
\bibliographystyle{ACM-Reference-Format}
\bibliography{paper}
\end{document}
