%% This is an abbreviated template from http://www.sigplan.org/Resources/Author/.

\documentclass[acmsmall,review,nonacm]{acmart}
\begin{document}

%%
%% The "title" command has an optional parameter,
%% allowing the author to define a "short title" to be used in page headers.
\title{State-Machine Qualia and it's effects on Context-Free Language Reachability}

%%
%% The "author" command and its associated commands are used to define
%% the authors and their affiliations.
%% Of note is the shared affiliation of the first two authors, and the
%% "authornote" and "authornotemark" commands
%% used to denote shared contribution to the research.
\author{Henri Malahieude}
\email{henri.malahieude@colorado.edu}
%%\author{TBD}
%%\email{tbd@colorado.edu}
\affiliation{%
  \institution{University of Colorado Boulder}
  \country{USA}
}

%%
%% The abstract is a short summary of the work to be presented in the
%% article.
\begin{abstract}
	Since Thomas Reps' paper, Program Analysis via Graph Reachability~\cite{reps_ilps_1997}, 
	much work has been devoted to optimizing the reachability solver~\cite{zhang_pldi_2013,lei_pldi_2024},
	refining the graph construction process~\cite{lei_pldi_2023},
	and finding the ordering of analysis questions with respect to graph reachability ~\cite{ding_sas_2023}.
	However, there has been no study yet, to our knowledge, of the types of analysis for which Context-Free Language Reachability (CFLR) cannot be deployed in
	the traditional techniques of a Dyck-Language and the Pushdown Automata Reachability Solver.
	%This work seeks to establish the limitations of CFLR for program analysis by drawing a relationship between the languages employed for each question, 
	%thus the state-machine required to compute the reachability answer, and the analysis question of interest.
	For example, dominance of one basic block on the next is not an analysis for which Dyck-Reachability has been employed for, nor is there has there been work for why that is.
	This work argues that with the current CFLR context, Dominance, and other analyses that require reasoning over multiple paths, represent a class of questions that cannot be solved.
	Furthermore, the Automata required to solve the problem can bring insight into the analysis of interest including a best approach to optimizing an algorithm.
\end{abstract}

%%
%% This command processes the author and affiliation and title
%% information and builds the first part of the formatted document.
\maketitle

\section{Introduction}
\label{sec:intro}
Context-Free Language Reachability, introduced by Thomas Reps~\cite{reps_ilps_1997}, 
is an approach to program analysis that leverages graph theory and context-free languages to achieve complex inter-procedural analysis.
In Context-Free Language Reachability (CFLR), each edge in a graph is annotated with a terminal from a Context-Free Language.
As a graph is traversed each terminal is concatenated together into a final string, 
such that only paths which can both be reached by the structure of the graph and produce an accepted word in the language are considered ``reachable''.
Using a simple Dyck-Language of balanced parentheses, inter-procedural analysis is no longer ambiguous: 
only paths calling function x, and therefore have $(_x$ in the string, may follow the return edge $)_x$.

However, this paper argues that CFLR is limited in the analyses that it can do.
Some analyses require reasoning over many paths, but CFLR's current formulation only considers \textit{the existence} of a single path between two vertices.
Other analyses do not have a clear ``beginning'' and ``end'' to represent with the open and closed parenthesis, which has been the traditional language for many CFLR analyses.
An analysis of these limitations, we argue, will help create a new approach for both generating a language for an analysis of interest and studying currently existing analyses.


\section{Overview}
\label{sec:over}
This paper is organized as follows:
\begin{itemize}

\item Section \ref{sec:example} introduces the example which exhibits both limitations for which CFLR does not yet solve for.
\item Section \ref{sec:ecflr} proposes an extension to CFLR to solve both limitations.
\item Section \ref{sec:automata} studies what this extension means for analyses using CFLR through the lens of automata theory.
\item Section \ref{sec:related} discusses related works.
\item Section \ref{sec:conclusion} summarizes the arguments laid out in this work.

\end{itemize}


\section{Motivating Example}
\label{sec:example}
Context-Free Language Reachability has six varieties~\cite{reps_ilps_1997}:
\begin{itemize}
\item All-Pairs, there exists a path between all pairs of nodes.
\item Single-Source, there exists a path between a selected source and all other nodes.
\item Single-Target, there exists a path between all other nodes to a selected node.
\item Single-Source\&Target, there exists a path between two specific nodes.
\item Multi-Source, there exists a path between multiple sources and all other nodes.
\item Multi-Target, there exists a path between all other nodes to multiple targets.
\end{itemize}

The first limitation of Context-Free Language Reachability (CFLR) is that some analyses require reasoning over multiple paths.
It is not just that we must know whether or not a path exists, but the analysis wants to know about the shared properties of these paths.
In all varieties listed above, the only property of interest is existence.
There is no reasoning about paths themselves.

The second limitation is that the traditional language for CFLR, the Dyck-Language of k numbers of balanced parentheses, only works for analyses that have clear starts and ends.
Analyses which are attempting to solve for those points, their input graph could not be annotated with a Dyck-Language without first solving for those start and end points beforehand.

\begin{figure}
	\includegraphics[width=0.25\textwidth]{figs/CFG_Example.pdf}
	\caption{A Control-Flow Graph for which traditional Context-Free Language Reachability cannot solve Dominance for. 
Basic Block A dominates E \& F, but B only dominates E. C does not dominate anything.}
	\label{fig:cfg}
\end{figure}

%Since CFLR is looking for the existence of a path between vertices of a graph, it is a powerful tool for demand-based analysis.
%But weak for analyses for which no known demand-based algorithm exists, such as Dominance.

Dominance Analysis on Control-Flow Graphs exhibits both of these limitations.
In Figure \ref{fig:cfg}, it is difficult to use parentheses to symbolize dominance. 
Assume that we could know where to place the symbols to denote the beginning of the dominance and the end of dominance for A.
For example, on the edges from A we can place $(_a$ and on the edges into F we can place $)_a$.
If their positions are already known, it follows that dominance is already calculated for A since we know the region at which it dominates and does not.
Annotating the graph for a separate solver to traverse it would then be redundant work.
This is a contradiction that means that we cannot use a balanced language for this analysis.

%Furthermore, there does not exist two edges to place the beginning and end of dominance for Basic Block B.
%Even if we decide to expand our language to also accept any unclosed $(_x$ as accepted words, and annotate all edges with their source parenthesis
%that would result in B dominating F which is not correct.

Applying transformations on the input graph may be the next possible step, but in what way?
The more processing steps done to the input graph before the final solver just bypasses the CFLR limitation by not using CFLR.


\section{Extended Context-Free Language Reachability}
\label{sec:ecflr}
TODO

\section{Automata and Analysis}
\label{sec:automata}
TODO

\section{Related Work}
\label{sec:related}
\subsection{Automatas and the Cubic Bottleneck}
The Cubic Bottleneck for program analysis has been a well-studied issue,
and was one of the topics discussed in Rep's original CFLR paper~\cite{reps_ilps_1997}.
Melski and Reps~\cite{melski_sigplan_1997}, and Kodumal and Aiken~\cite{kodumal_pldi_2004}, 
study this issue from an conversion perspective between two analysis techniques, Set-Constraints and CFLR.
Heintze and McAllester~\cite{heintze_lics_1997} approach the topic from a similar angle as this paper, using automata theory. 
They argue that almost all analysis questions can be converted into 2-way Non-deterministic Pushdown Automata (2NPDA), 
and since the late 1960s there has not been a sub-cubic algorithm for computing 2NPDAs.

%Improved automatas for CFLR have been proposed by Alur et al.~\cite{alur_stoc_2004,alur_toplas_2005},
%focused on optimizing for Dyck-Languages and the pushdown concept respectively.
Visibly Pushdown Languages~\cite{alur_stoc_2004}, proposed by Alur and Madhusudan, 
focus on dividing the automata's alphabet ($\Sigma$) into call symbols, return symbols, and internal symbols.
Unless there is a call or return symbol, the stack is not used. 
This allows usefull properties, such as determining whether a language is included in another for two Visibly Pushdown Automatas (VPAs).
However, VPAs do not have a method of improving on the cubic bottleneck.

Recursive State-Machines~\cite{alur_toplas_2005}, also proposed by Alur et al., 
encode memory into the path of the automata rather that in a separate stack.
States of a Recursive State-Machine can be state-machines as well as states.
The improvements to traditional automata are similar to VPAs, 
however they crucially provide the possibility for sub-cubic analysis.

\subsection{Context-Free Language Reachability}
Various developments in the field of Context-Free Language Reachability have taken place in the last couple of years.
Zhang et al.~\cite{zhang_pldi_2013} exploit an equivalence property in bidirectional graphs for faster alias analysis.
Ding et al.~\cite{ding_sas_2023} have formulated a method for composing multiple analyses into a single pass, 
or dividing a single analysis into multiple pieces that can allow for faster exits.
Lei et al.~\cite{lei_pldi_2023,lei_pldi_2024} propose methods of optimizing the input graph representing the analysis to speed up the solver.




\section{Conclusion}
\label{sec:conclusion}
TODO

%%
%% The acknowledgments section is defined using the "acks" environment
%% (and NOT an unnumbered section). This ensures the proper
%% identification of the section in the article metadata, and the
%% consistent spelling of the heading.
\begin{acks}
	TODO
\end{acks}

%%
%% The next two lines define the bibliography style to be used, and
%% the bibliography file.
\bibliographystyle{ACM-Reference-Format}
\bibliography{paper}
\end{document}
