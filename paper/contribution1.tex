The six previously discussed variations of CFLR in Section \ref{sec:example} are quite powerful for demand-based analysis.
By only computing for the existence of a path on nodes of interest, refined answers can be produced with less time than a holistic approach.

For Dominance however they remain insufficient, as the analysis needs to compute all possible paths between two nodes and then find similarities in those paths.
To solve the presented issue, a new type of Context-Free Language Reachability (CFLR) analysis is defined as follows.
An \textit{All-Paths} variation on Context-Free Language Reachability is a solver divided into two parts, the reachable paths engine, and the paths analyzer.

\begin{figure}

\includegraphics[width=0.25\textwidth]{figs/CFG_Example2.pdf}
\caption{The CFG from the previous example, where each Basic Block is annotated with the symbol that corresponds to its source Block.}
\label{fig:a-cfg} %annotated cfg

\end{figure}

Applying the All-Paths, and Single-Source selecting node A, variation to Dominance Analysis, we can provide an almost trivial language for CFLR.
In Figure \ref{fig:a-cfg}, each edge is labeled with its source node's name, and all words are accepted for our example Control-Flow Graph.
Thus the word is now representing memory as well.

For our example CFG in Fig. \ref{fig:a-cfg} all possible words between A and F are:
\begin{itemize}

\item ac
\item abe
\item abde

\end{itemize}

Our final pass on the paths would be a meet operator which produces all terminals of which are the same between the words.
Therefore A dominates F, and we have an algorithm using the CFLR technique to calculate the strict dominance of each node. 
Dominance is then the simple union of the final calculation with the target node.
